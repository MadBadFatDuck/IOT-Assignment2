\documentclass[a4paper,12pt]{article}

% Pacchetti per la lingua e la codifica
\usepackage[utf8]{inputenc}
\usepackage[T1]{fontenc}
\usepackage[italian]{babel}

% Pacchetti per la grafica e le immagini
\usepackage{graphicx}
\usepackage{float}
\usepackage{caption}
\usepackage{subcaption}

% Pacchetti per la formattazione del codice
\usepackage{listings}
\usepackage{xcolor}

% Pacchetti per tabelle e layout
\usepackage{geometry}
\usepackage{booktabs}
\usepackage{hyperref}

% Configurazione margini
\geometry{top=2.5cm, bottom=2.5cm, left=2.5cm, right=2.5cm}

% Configurazione colori per il codice
\definecolor{codegreen}{rgb}{0,0.6,0}
\definecolor{codegray}{rgb}{0.5,0.5,0.5}
\definecolor{codepurple}{rgb}{0.58,0,0.82}
\definecolor{backcolour}{rgb}{0.95,0.95,0.92}

\lstdefinestyle{mystyle}{
    backgroundcolor=\color{backcolour},   
    commentstyle=\color{codegreen},
    keywordstyle=\color{magenta},
    numberstyle=\tiny\color{codegray},
    stringstyle=\color{codepurple},
    basicstyle=\ttfamily\footnotesize,
    breakatwhitespace=false,         
    breaklines=true,                 
    captionpos=b,                    
    keepspaces=true,                 
    numbers=left,                    
    numbersep=5pt,                  
    showspaces=false,                
    showstringspaces=false,
    showtabs=false,                  
    tabsize=2
}

\lstset{style=mystyle}

% Intestazione documento
\title{\textbf{Relazione di Progetto: Smart Drone Hangar}\\
\large Assignment 2 - IoT Systems}
\author{Karim El Berni \and Luca Fabbri \and Luca Dellasantina}
\date{\today}

\begin{document}

\maketitle
\tableofcontents
\newpage

\section{Introduzione}
Il progetto \textbf{Smart Drone Hangar} consiste nella realizzazione di un sistema IoT per la gestione automatizzata di un hangar per droni. Il sistema è composto da due sottosistemi principali che comunicano tramite protocollo seriale:
\begin{itemize}
    \item \textbf{Drone Hangar (Arduino):} L'unità fisica che gestisce sensori (temperatura, distanza, presenza) e attuatori (motori, LED, display LCD) per controllare l'accesso e lo stato dell'hangar.
    \item \textbf{Drone Remote Unit (Java/PC):} Un'interfaccia grafica (GUI) che funge da dashboard di controllo remoto, permettendo all'operatore di inviare comandi e monitorare la telemetria in tempo reale.
\end{itemize}

\section{Architettura del Sistema}
\begin{figure}[h]
    \centering
    \includegraphics[width=0.8\textwidth]{Drone_hangar.png}
    \caption{Schema del circuito su Breadboard}
    \label{fig:circuit}
\end{figure}

\subsection{Hardware - Lato Arduino}
Il sottosistema embedded è basato su Arduino Uno e utilizza i seguenti componenti collegati ai pin specificati:

\begin{table}[h]
\centering
\begin{tabular}{|l|l|l|}
\hline
\textbf{Componente} & \textbf{Pin Arduino} & \textbf{Funzione} \\
\hline
Led L1 (Verde) & D2 & Stato "Power On" / "Rest" \\
Led L2 (Verde) & D3 & Indicatore di movimento (Blinking) \\
Led L3 (Rosso) & D4 & Indicatore di Allarme \\
Servo Motore & D5 & Apertura/Chiusura portellone \\
Sonar (HC-SR04) & D9 (Trig), D10 (Echo) & Misurazione distanza drone \\
Sensore PIR & D11 & Rilevamento presenza drone esterno \\
Pulsante & D12 & Reset manuale allarme \\
Sensore Temp (TMP36) & A0 & Monitoraggio surriscaldamento \\
Display LCD (I2C) & SDA/SCL & Feedback visuale all'utente \\
\hline
\end{tabular}
\caption{Mappa dei collegamenti hardware}
\end{table}

\subsection{Software - Lato PC}
Il lato PC è un'applicazione Java basata su \textbf{Swing} per l'interfaccia grafica. La comunicazione seriale è gestita tramite la libreria \texttt{jSerialComm}, che permette l'invio asincrono di comandi e la ricezione di eventi.

\section{Implementazione Firmware (Arduino)}

\subsection{Architettura a Task}
Il firmware è stato progettato seguendo un pattern basato su uno \textbf{Scheduler Cooperativo}. Invece di utilizzare un unico ciclo \texttt{loop()} bloccante o complesso, il sistema è diviso in task indipendenti:
\begin{itemize}
    \item \textbf{Scheduler:} Gestisce l'esecuzione periodica dei task.
    \item \textbf{SmartHangarTask:} Implementa la logica principale e la macchina a stati finiti (FSM).
    \item \textbf{BlinkTask:} Gestisce il lampeggio dei LED in modo non bloccante.
    \item \textbf{TempMonitorTask:} Monitora la temperatura e innesca gli allarmi.
\end{itemize}

\subsection{Macchina a Stati Finiti (FSM)}
Il cuore del sistema è la classe \texttt{SmartHangarTask}, che evolve attraverso i seguenti stati:

\begin{enumerate}
    \item \textbf{REST:} Il sistema è a riposo, il drone è dentro. Il servo è chiuso (0°). Attende il comando seriale "TO".
    \item \textbf{TAKE\_OFF:} Il servo si apre (90°). Il sistema attende che il Sonar rilevi che il drone è uscito ($distanza > D1$).
    \item \textbf{TO\_OUT:} Il drone è fuori e il servo si richiude. Attende il comando "LD" E il rilevamento del PIR.
    \item \textbf{LANDING:} Il servo si riapre. Attende che il Sonar rilevi il rientro del drone ($distanza < D2$).
    \item \textbf{ALARM:} Stato di emergenza attivato da temperatura elevata. Blocca le operazioni finché non viene premuto il pulsante di Reset.
\end{enumerate}

Di seguito un esempio del codice della FSM nel metodo \texttt{tick()}:

\begin{lstlisting}[language=C++]
void SmartHangarTask::tick() {
  if (state == REST) {
    if (Serial.available() > 0) {
      String cmd = Serial.readStringUntil('\n');
      if (cmd == "TO" && !preAlarm) {
        state = TAKE_OFF;
        servo->setPosition(90);
        // ...
      }
    }
  } else if (state == TAKE_OFF) {
    if (sonar->getDistance() > D1) {
        // Logica transizione a TO_OUT
    }
  }
  // Altri stati...
}
\end{lstlisting}

\subsection{Monitoraggio Temperatura}
Il task \texttt{TempMonitorTask} legge periodicamente il sensore TMP36. Gestisce due soglie:
\begin{itemize}
    \item \textbf{Temp1:} Attiva un flag di pre-allarme.
    \item \textbf{Temp2:} Se superata per un tempo $T4$, porta il sistema nello stato globale di \textbf{ALARM}, accendendo il Led L3 e bloccando le operazioni.
\end{itemize}

\section{Implementazione Software (Java)}

La \textbf{Drone Remote Unit} fornisce una GUI semplice con:
\begin{itemize}
    \item Due pulsanti per i comandi: \textit{Take Off} e \textit{Land}.
    \item Label per visualizzare lo stato corrente, la distanza rilevata e la temperatura.
\end{itemize}

La classe \texttt{SerialComm} gestisce la comunicazione su un thread separato per non bloccare l'interfaccia utente (EDT).

\begin{lstlisting}[language=Java]
// Esempio di gestione eventi seriali in Java
serialPort.addDataListener(new SerialPortDataListener() {
   @Override
   public void serialEvent(SerialPortEvent event) {
       // Lettura dati e aggiornamento GUI tramite SwingUtilities.invokeLater
   }
});
\end{lstlisting}

\section{Protocollo di Comunicazione}
Arduino e PC comunicano tramite messaggi ASCII terminati da \texttt{\textbackslash n} a 9600 baud.

\begin{table}[h]
\centering
\begin{tabular}{|l|l|l|}
\hline
\textbf{Direzione} & \textbf{Messaggio} & \textbf{Significato} \\
\hline
PC $\rightarrow$ Arduino & \texttt{TO} & Richiesta di decollo (Take Off) \\
PC $\rightarrow$ Arduino & \texttt{LD} & Richiesta di atterraggio (Land) \\
\hline
Arduino $\rightarrow$ PC & \texttt{TAKE OFF} & Conferma inizio decollo \\
Arduino $\rightarrow$ PC & \texttt{DRONE OUT} & Drone uscito dall'hangar \\
Arduino $\rightarrow$ PC & \texttt{LANDING} & Inizio procedura atterraggio \\
Arduino $\rightarrow$ PC & \texttt{DRONE INSIDE} & Drone rientrato (Stato REST) \\
Arduino $\rightarrow$ PC & \texttt{ALARM} & Allarme temperatura critica \\
Arduino $\rightarrow$ PC & \texttt{TEMP:xx.x} & Telemetria Temperatura \\
\hline
\end{tabular}
\caption{Protocollo Seriale}
\end{table}

\section{Conclusioni}
Il progetto dimostra l'integrazione efficace tra un sistema embedded real-time (Arduino con scheduler cooperativo) e un sistema di controllo desktop (Java). L'uso di una macchina a stati finiti garantisce robustezza nella gestione logica dell'hangar, prevenendo stati inconsistenti, mentre il monitoraggio della temperatura aggiunge un livello di sicurezza fondamentale per un sistema IoT.

\end{document}